%% exec_summ.tex  Executive  Summary John

sPHENIX\cite{sPHENIX:2015aa} is a proposal for a major upgrade
to the PHENIX experiment at RHIC capable of
measuring jets, photons, and Upsilon states to determine the
temperature dependence of transport coefficients of the quark-gluon plasma.
The detector needed to make these measurements requires electromagnetic and
hadronic calorimetry for measurements of jets, a high resolution and low mass
tracking system for reconstruction of the Upsilon states,
and a high speed data acquisition system.

This document describes the baseline design for a detector capable of carrying
out this program of measurements built around the BaBar solenoid.
As much as possible, the mechanical, electrical, and electronic infrastructure
developed for the PHENIX experiment from 1992-2016 is reused for sPHENIX.
The major new systems are the superconducting magnet, a high precision tracking system, and electromagnetic and hadronic calorimeters.

The central tracking system 
consists of a small Time Projection Chamber (TPC) with
three layers of Monolithic Active Pixel Sensors (MAPS) vertex detectors,
and two layers of silicon strip detectors within the inner radius. 

The electromagnetic calorimeter is a compact tungsten-scintillating fiber
design located inside the solenoid.  
The outer hadronic calorimeter consists of steel in which 
scintillator tiles with light collected by wavelength shifting fibers are sandwiched
between tapered absorber plates that project nearly radially from the interaction point.
The calorimeters use a common set of silicon photomultiplier photodetectors and
amplifier and digitizer electronics.

The baseline detector consists of the compact TPC and its readout electronics, an electromagnetic calorimeter with acceptance $|\eta| < 0.85$, an outer hadronic
calorimeter which doubles as the flux return of the solenoid, readout
electronics for the calorimeters, and data acquisition and trigger hardware.
In addition to the baseline detector proposed here, several complementary projects
have been pursued with alternative funding sources for the superconducting
magnet and support structure, the remaining 25\% of the electromagnetic 
calorimeter acceptance, an inner hadronic calorimeter, and two inner silicon detectors which both make the
tracking more robust and enable a heavy flavor physics program.

The detector design has been evaluated by means of GEANT4 simulation and a program of bench and beam tests of prototype detectors.
This program of simulation, prototyping, and testing of components has been used to converge on the baseline design described herein.
